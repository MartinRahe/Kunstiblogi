\documentclass[12pt,a4paper]{article}
\usepackage[utf8]{inputenc}
\usepackage[hmargin={1cm,1cm},vmargin={2cm,1.5cm}]{geometry}
\usepackage{amsmath}
\usepackage{siunitx}
\usepackage{titlesec}
\usepackage{graphicx}
%\usepackage[thinc]{esdiff}
\usepackage{ragged2e}
\usepackage{amsfonts}
\usepackage{amssymb}
\usepackage{titling}
\usepackage{mathtools}
\usepackage{fancyhdr}
\usepackage{polyglossia}
\usepackage{csquotes}
\setdefaultlanguage{estonian}
\usepackage[ddmmyyyy]{datetime}

\pagestyle{fancy}
\fancyhf{}
%\lhead{Kunstiajaloo KT}
%\rhead{\theauthor}
\renewcommand{\headrulewidth}{0pt}
\renewcommand{\footrulewidth}{0pt}

\DeclarePairedDelimiter\ceil{\lceil}{\rceil}
\DeclarePairedDelimiter\floor{\lfloor}{\rfloor}

\let\oldepmtyset\emptyset
\let\emptyset\varnothing

\renewcommand{\baselinestretch}{1.5}
\setlength{\parindent}{0cm}
\setlength{\parskip}{0.3cm}
\tolerance=1000000
\renewcommand{\thesection}{\normalsize{\arabic{section})}}
\renewcommand{\thesubsection}{\large{\arabic{subsection})}}
\renewcommand{\thesubsubsection}{\normalsize{\arabic{subsection}.\arabic{subsubsection})}}
\renewcommand{\dateseparator}{.}

%\pretitle{\begin{flushleft}\LARGE} % makes document title flush right
	%\posttitle{\end{flushleft}}
%\preauthor{\begin{flushleft}\large} % makes author flush right
	%\postauthor{\end{flushleft}}
%\predate{\begin{flushleft}\large} % makes date title flush right
	%\postdate{\end{flushleft}}

\title{Kunstinäituse retsensioon}
\author{Martin Rahe}
\date{16.09.2020}

\begin{document}
	\maketitle
	\thispagestyle{fancy}
	
	Külastasin Eesti Arhitektuurimuuseumi püsinäitust \enquote{ELAV RUUM: sajand Eesti arhitektuuri}. Näituse kuraatorid on Mait Väljas ja Carl-Dag Lige. Näitus annab ülevaate Eesti arhitektuuri muutustest 20. sajandil. Näitusel on esindatud maketid paljudest Eesti arhitektuuri kujunemises olulistest ehitistest. Iga maketi juurde on lisatud infokaart, mis aitab paremini mõista teost ja autori inspiratsiooni. Mulle see näitus meeldis, oli huvitav mõtiskleda arhitektuuri arengu üle ning seostada seda ühiskondlikke protsessidega. Valisin külastamiseks just selle näituse, kuna arhitektuur on minu jaoks huvitavam kui teised kunstiliigid.
	
	Kõige rohkem pakkus mulle huvi arhitekt Robert Natuse poolt projekteeritud Tallinnas Roosikrantsi 23 / Pärnu mnt 36 asuv elamu ja ärihoone, mis valmis 1936. aastal. Robert Natus oli esimene Eestis arhitektikutse saanud arhitekt. Koos Natuse teise klinkermajaga (praegune Tallinna linnavalitsuse hoone Vabaduse väljaku ääres) on see Saksa tellisekspressionismist inspireeritud ehitis üks Natuse esiletõstetumaid teoseid. Ühtlasi on need kaks hoonet kõige otsesemad ekspressionismi näited Eesti arhitektuuris. Aastal 1997 kuulutati see hoone ehitismälestiseks.
	
	Lisaks Tallinnas tavatule materjalivalikule on selle hoone juures tähelepanuväärne selle krundi äärmiselt terav tipp ning sellest tulenev hoone iseäralik kuju, seetõttu paneb teos mind mõtlema arhitektuuri mitmekülgsusele. Hoone suured mõõtmed, tume toon ja ekspressionistlik stiil jätavad vaatajale kindlasti võimsa mulje. Hoone on ehitatud klinkertellistest ja on ühtlaselt tumepruuni tooni, detailseks teeb hoone selle fassaadi kolmemõõtmelisus. Hoones on kasutatud tellisekspressionismile omaseid skulptuurseid detaile, näiteks paikneb hoone nurgas, uste vahel tagajalgadel seisev lõvifiguur, mis hoiab vapikilpi ehitusaastaga. Olulised on ka hoone fassaadtiibadel asuvad etteasted, mis lisavad hoone iseloomulikule välimusele. Hoonet ennast on tavaolukorras võimalik näha vaid alt üles või otse vaadates, kusjuures erinevast kohast näeb veidi erinevat külge. Maketilt on näha, et ka ülaltvaates on hoone kuju huvitav ja katusel on näha detaile, mida alt vaadates on seinte poolt osaliselt varjatud.
	
	Hoone ehitati Eesti vaikiva ajastu perioodil, mil rahvuslus oli olulisel kohal. Seetõttu on võimalik, et inspiratsiooni võtmine Saksa arhitektuurist oli osaliselt tingitud soovist samastuda sakslastega ja kaugeneda Venemaast, kuid minu hinnangul on see ebatõenäoline ja pigem põhjustas sellise valiku lihtsalt arhitekti isiklik stiilieelistus. 
	
	\begin{figure}[h!]
		\includegraphics[height=0.4\textheight]{makett1}
	\end{figure}

	\begin{figure}[h!]
		\includegraphics[height=0.4\textheight]{makett2}
	\end{figure}
\end{document}